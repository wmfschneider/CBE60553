% Created 2017-09-12 Tue 21:38
% Intended LaTeX compiler: pdflatex
\documentclass[11pt]{article}
\usepackage[utf8]{inputenc}
\usepackage{lmodern}
\usepackage[T1]{fontenc}
\usepackage{fixltx2e}
\usepackage{graphicx}
\usepackage{longtable}
\usepackage{float}
\usepackage{wrapfig}
\usepackage{rotating}
\usepackage[normalem]{ulem}
\usepackage{amsmath}
\usepackage{textcomp}
\usepackage{marvosym}
\usepackage{wasysym}
\usepackage{amssymb}
\usepackage{amsmath}
\usepackage[version=3]{mhchem}
\usepackage[numbers,super,sort&compress]{natbib}
\usepackage{natmove}
\usepackage{url}
\usepackage{minted}
\usepackage{underscore}
\usepackage[linktocpage,pdfstartview=FitH,colorlinks,
linkcolor=blue,anchorcolor=blue,
citecolor=blue,filecolor=blue,menucolor=blue,urlcolor=blue]{hyperref}
\usepackage{attachfile}
\usepackage[left=1in, right=1in, top=1in, bottom=1in, nohead]{geometry}
\geometry{margin=1.0in}
\usepackage{hyperref}
\usepackage{amsmath}
\usepackage{graphicx}
\usepackage{epstopdf}
\usepackage{fancyhdr}
\pagestyle{fancy}
\fancyhf{}
\usepackage[labelfont=bf]{caption}
\usepackage{setspace}
\setlength{\headheight}{10.2pt}
\setlength{\headsep}{20pt}
\renewcommand{\headrulewidth}{0.5pt}
\renewcommand{\footrulewidth}{0.5pt}
\lfoot{\today}
\cfoot{\copyright\ 2016 W.\ F.\ Schneider}
\rfoot{\thepage}
\chead{\bf{Advanced Chemical Engineering Thermodynamics (CBE 60553)\vspace{12pt}}}
\lhead{\bf{Homework 3}}
\rhead{\bf{Due September 22, 2017}}
\usepackage{titlesec}
\titlespacing*{\section}
{0pt}{0.6\baselineskip}{0.2\baselineskip}
\title{University of Notre Dame\\Advanced Chemical Engineering Thermodynamics\\(CBE 60553)}
\author{Prof. William F.\ Schneider}
\usepackage{siunitx}
\usepackage[version=3]{mhchem}
\def\dbar{{\mathchar'26\mkern-12mu d}}
\setcounter{secnumdepth}{3}
\author{William F. Schneider}
\date{\today}
\title{CBE 60553 Homework}
\begin{document}

\begin{OPTIONS}
\end{OPTIONS}

\noindent \textbf{Solve each problem on separate sheets of paper, and clearly indicate the problem number and your name on each.  Carefully and neatly document your answers.  You may use a mathematical solver like Python. Use plotting software for all plots.}

\section{Shut that door!}
\label{sec:org310cd53}
A household refrigerator is kept at  \(2^\circ\text{C}\).  Every time the door is open
   to put something inside, \SI{200}{kJ} of heat are added to the refrigerator and the
   temperature inside changes negligibly.  The door is opened 15 times a day, and the
   refrigerator operates at 15\% of theoretical maximum efficiency.  The cost of
   electricity is \$0.10/kW h.

\begin{enumerate}
\item Derive an expression for the theoretical maximum efficiency of the refrigerator.

\item How much does it cost to run the refrigerator for a month?
\end{enumerate}

\section{\href{https://open.spotify.com/track/5JJDu0Z5DKe7mR31MGksSg}{Gotta keep them separated}}
\label{sec:org5a42589}
A geothermal source is to be used to power a plant that separates oxygen from
  air. The geothermal source comprises a well containing \(10^3~\text{m}^3\) of
  water initially at \(100^\circ\text{C}\); nearby there is a very big (\(>>
  10^3~\text{m}^3\) ) lake at \(5^\circ\text{C}\).

\begin{enumerate}
\item What is the theoretical maximum amount of work that could be extracted from the power source?
Assume the heat capacity of \ce{H2O} is \(4.2~\text{J}~\text{(g K)}^{-1}\), independent of temperature.
\end{enumerate}

\noindent Suppose you'd like to use the work from the geothermal source to
  extract oxygen from air.  The atmosphere is a comfy \(20^\circ\text{C}\) and can
  be treated as an ideal gas mixture of 80\% \ce{N2} and 20\% \ce{O2} at 1 atm.
  The fundamental equation of an ideal gas mixture is like that of an ideal gas,
  with the addition of terms related to the mixture composition. It's convenient
  to write the composition in terms of mole fraction, \(y_{j} = N_{j}/N\), where
  \(N = \sum_{j}N_{j}\).  The fundamental equation can be written

\begin{equation}
S(T,P,N_{1},N_{2},\ldots) = c R \ln\left ( \frac{T}{T_{0}} \right ) - R \ln\left ( \frac{P}{P_{0}} \right ) - R \sum_{j}y_{j}\ln y_{j} + \sum_{j}y_{j}s_{0,j}
\end{equation}

\noindent where \(P = NRT/V\) and \(U=cNRT\).

\begin{enumerate}
\item What is an appropriate value for \(c\)?

\item What is the maximum number of moles of \ce{O2} that could be produced from the geothermal source?  Assume the separation is carried out at constant temperature and pressure.

\item Propose some plausible way that the separation could be carried out.
\end{enumerate}

\section{Fundamentally van der Waals}
\label{sec:org57e100b}
The fundamental equation of a ``van der Waals'' fluid is:
\begin{equation}
s_\text{vdW}(u,v)=s_{0}+R\ln\left (v-b\right ) +c R \ln \left ( u+a/v \right )
\end{equation}

\begin{enumerate}
\item Derive the equations of state of a van der Waals fluid.

\item Look up the critical constants and (constant volume) heat capacity of \ce{CO2}.  Use those to determine appropriate values for \(a\), \(b\), and \(c\) of van der Waals \ce{CO2}.  Remember to include appropriate units.

\item Plot on the same graph the \(P-v\) adiabats and isotherms of van der Waals \ce{CO2}.  You will see in the extended course notes some python code that might help.

\item Suppose you have a gas with a heat capacity appropriate to \ce{CO2} at \SI{350}{K} and \SI{50}{bar}. Compute the work required and the final state (temperature, pressure, and molar volume) to

\begin{enumerate}
\item Isothermally compress to \SI{200}{bar}, assuming the gas is ideal.

\item Adiabatically compress to \SI{200}{bar}, assuming the gas is ideal.

\item Isothermally compress to \SI{200}{bar}, assuming the gas is a van der Waals fluid.

\item Adiabatically compress to \SI{200}{bar}, assuming the gas is a van der Waals fluid.
\end{enumerate}
\end{enumerate}


\section{I'm (Legendre) transformed}
\label{sec:org77c8c6b}
Consider the exponential function \(y=2 e^{x/2}\).

\begin{enumerate}
\item Construct the Legendre transform of \(y\).  Let's call it \(\phi(m) = y - m x\), where \(m = dy/dx\).

\item Use software to plot the series of lines defined by \(\phi(m)\). Recall that \(\phi(m)\) is
the intercept of a line of slope \(m\).  What  function does the series of lines trace out?

\item A Legendre transform can be undone by performing an ``inverse'' transform.  The inverse
Legendre transform is constructed in the same way as the transform itself,
\emph{except} that \(m=-dy/dx\). Perform an inverse Legendre transform on \(\phi(m)\).
\end{enumerate}
\end{document}